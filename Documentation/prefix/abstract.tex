%\begingroup
%\addchap{Kurzfassung} 
%\endgroup

{\huge Kurzfassung}
\\
\\

Die Arbeit befasst sich mit der Analyse von Sequenzen, die Zustände über einen zeitlichen Verlauf darstellen.
Zu Beginn der Arbeit werden diese formal eingeführt und ihre Eigenschaften definiert.
Auf Basis dieser Grundlagen werden binäre Sequenzen abgeleitet.
Für diese wird ein Verfahren zur Analyse vorgestellt.\\
In der Regel handelt es sich bei Daten von Produktionslinien nicht um binäre Sequenzen, sondern um maschinen-spezifische Daten, die mehrere Zustände annehmen können.
Die Menge der mögliche Zustände muss auf ein in der Arbeit entwickeltes Format konvertiert werden.
Dabei findet eine Abbildung der Zustände auf eine binären Wertebereich statt.
Ein entsprechendes Tool wurde im Zuge der Arbeit entwickelt.\\
Die konvertierten Daten werden von dem im Voraus genannten Verfahren analysiert.
Dieses wurde durch automatisierte und anpassbare Skripte umgesetzt und wird in dieser Arbeit vorgestellt. 
Dabei werden PDF-Dateien generiert, die zu weiteren Analysen verwendet werden können.
\newpage

%\begingroup
%\addchap{Abstract} 
%\endgroup

{\huge Abstract}
\\
\\

The work deals with the analysis of sequences representing states over time.
At the beginning of the work, these are formally introduced and their properties defined.
Binary sequences are derived on the basis of these fundamentals.
For these, a method for analysis is presented.\\
Generally, data from production lines are not binary, but machine-specific data that can assume several states.
The set of possible states must be converted to a format developed in the thesis.
The states are mapped to a binary value range.
A corresponding tool was developed in the course of the work.\\
The converted data are analyzed by the previously developed procedure.
This was implemented by automated and adaptable scripts and is presented in this thesis.
PDF files are generated, which can be used for further analysis.

\newpage