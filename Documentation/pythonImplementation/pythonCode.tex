\chapter{Implementierungen in Python}

\section{Warum Python?}
Python ist eine Skriptsprache, die auch im wissenschaftlichen Umfeld und für maschinelles Lernen verwendet wird\footnote{Siehe: \cite{Fuxjaeger2017}.}.
Dadurch existieren im Umfeld der Programmiersprache viele Bibliotheken die eingebunden und verwendet werden können.
Aufgrund der großen Nutzerbasis und Dokumentation der verwendeten Bibliotheken sind notwendige Informationen leichter zu recherchieren.
Ein weiterer Vorteil von Python kommt daher, dass es sich um eine Skriptsprache handelt, die interpretiert wird.
Anpassungen können schnell umgesetzt und getestet werden, ohne die komplette Anwendung neu zu kompilieren und zu veröffentlichen.

\section{Verwendete Bibliotheken}
Zwei externe Bibliotheken werden für das Projekt benötigt:
\begin{description}[style=nextline]
	\item[NumPy] Eine Bibliothek für mathematische Operationen auf multi-dimensionalen Arrays\footnote{Siehe: \cite{Athanasias2014}.}. Aus dieser Bibliothek werden mathematische Funktionen verwendet.
	\item[Matplotlib] Diese Bibliothek bietet Plotting-Funktionalitäten zur Darstellung von Graphiken\footnote{Siehe: \cite{Matplotlib2019}.}. Sie wird verwendet um PDF-Dateien aus der Analyse zu generieren.
\end{description}
\section{Installationsanleitung}
Damit die Skripte lauffähig sind, müssen sowohl Python als auch die zusätzlichen Bibliotheken installiert werden.
Benötigt wird Python 3 (in der Entwicklung wurde Version 3.7.0 verwendet).
Die zusätzlichen Bibliotheken können über \enquote{pip}\footnote{Siehe: \cite{Foundation2019}.}, das Paketverwaltungsprogramm für Python, installiert werden. 
Folgende Befehle können zur Installation der Paketen verwendet werden.

\subsection{Paketinstallation unter Windows}
Unter Windows können die Pakete über das folgende Kommando installiert werden (vorausgesetzt, Python 3 ist installiert. Pip sollte bei der Installation von Python bereits installiert werden):
\lstnewenvironment{PipCommandWindows}
{\lstset{
		captionpos=b,
		frame=single,
		label={lst:pipwindows},
		caption={Kommando zur Installation der notwendigen Pakete unter Windows},
		showspaces=false,
		showtabs=false,
		breaklines=true,
		showstringspaces=false,
		breakatwhitespace=true,
		escapeinside={(*@}{@*)},
		commentstyle=\color{greencomments},
		morecomment = [l]{\#\ },
		morekeywords={ },
		keywordstyle=\color{blue},
		stringstyle=\color{black},
		morestring=[b]",
		basicstyle=\ttfamily\footnotesize,
		numberbychapter=false}}{}

\begin{PipCommandWindows}
	pip install numpy matplotlib
\end{PipCommandWindows}

\subsection{Paketinstallation unter Linux}
Unter Linux kann es notwendig werden, die Pakete explizit für Python 3 zu installieren. Im Folgenden sind die dafür notwendigen Kommandos abgebildet.
\lstnewenvironment{PipCommandLinux}
{\lstset{
		captionpos=b,
		frame=single,
		label={lst:piplinux},
		caption={Kommando zur Installation der notwendigen Pakete unter Linux},
		showspaces=false,
		showtabs=false,
		breaklines=true,
		showstringspaces=false,
		breakatwhitespace=true,
		escapeinside={(*@}{@*)},
		commentstyle=\color{greencomments},
		morecomment = [l]{\#\ },
		morekeywords={ },
		keywordstyle=\color{blue},
		stringstyle=\color{black},
		morestring=[b]",
		basicstyle=\ttfamily\footnotesize,
		numberbychapter=false}}{}

\begin{PipCommandLinux}
	pip install numpy matplotlib
\end{PipCommandLinux}
Hinweis: Je nach Konfiguration des Systems kann auch folgendes Kommando notwendig sein, um die Pakete explizit unter Python 3 zu installieren: 

\lstnewenvironment{Pip3CommandLinux}
{\lstset{
		captionpos=b,
		frame=single,
		label={lst:pip3linux},
		caption={Kommando zur Installation der notwendigen Pakete unter Linux (Verwendung von pip3)},
		showspaces=false,
		showtabs=false,
		breaklines=true,
		showstringspaces=false,
		breakatwhitespace=true,
		escapeinside={(*@}{@*)},
		commentstyle=\color{greencomments},
		morecomment = [l]{\#\ },
		morekeywords={ },
		keywordstyle=\color{blue},
		stringstyle=\color{black},
		morestring=[b]",
		basicstyle=\ttfamily\footnotesize,
		numberbychapter=false}}{}

\begin{Pip3CommandLinux}
	pip3 install numpy matplotlib
\end{Pip3CommandLinux}

\section{Verwendung von Docker}
\textcolor{red}{TODO}

\section{Aufbau des Python Projekts}
Der Quellcode ist nach der jeweiligen Funktionalität in eigene Module aufgeteilt.
Dadurch ist es auch möglich, nur einzelne Funktionalitäten in eigenen Skripten zu verwenden.
Die Module selbst sind wiederum in einzelne Dateien gegliedert.
Dadurch werden die Aufgaben und Funktionalitäten noch weiter aufgeteilt, um die Lesbarkeit des Codes zu erhöhen und die Komplexität zu reduzieren.

\subsection{Top-Level Ansicht}
Im Oberverzeichnis befinden sich die folgenden Dateien und Ordner. Dieses Verzeichnis dient als Einstiegspunkt zur Ausführung der Skripte.
\begin{description}[style=nextline]
	\item[Skriptdateien] Der Einstiegspunkt zur automatisierten Ausführung bildet das Skript in \enquote{automated\_script.py}. Zum Zweck der Dokumentation sind weitere Beispielskripte abgelegt, die das Prefix \enquote{example} im Dateinamen tragen.
	\item[Unterordner] In den Unterordnern \enquote{analysationrequest} \enquote{categorization} und \enquote{crosscorrelation} sind die Skripte für den jeweiligen Bereich abgelegt. Diese werden in den folgenden Sektionen genauer beschrieben. Eine Ausnahme bildet der \enquote{sourceFiles} Ordner. Dieser Ordner wird für die Quelldateien verwendet, wenn bei der Ausführung kein alternativer Pfad angegeben wird. Sollte der Ordner nicht existieren, so wird dieser automatisch erstellt.
	\item[Docker-File] Diese Datei kann verwendet werden, um einen Docker-Container zu erstellen. Siehe dazu die notwendigen Befehle im Anhang unter ~\ref{docker:examples}.
	\item[Dependencies.txt] In dieser Datei ist der Befehl zur Installation der notwendigen Pakete hinterlegt.
\end{description}

\subsection{Analysation-Request}
Das Modul \enquote{AnalysationRequest} bildet die Grundlage der Analysen.
Für jede Quelldatei wird ein Request angelegt, der anschließend durch die weiteren Module verarbeitet wird.
Die Ergebnisse der jeweiligen Module werden ebenfalls im Request festgehalten.
Dieses Modul beinhaltet auch die Logik und Funktionalität zum Einlesen der Quelldateien, sowie zur Expansion der komprimierten Quelldaten.
Eingelesen werden können Dateien die dem in Kapitel~\ref{csvMoreFiles} beschriebenen komprimierten Datenformat entsprechen.

\subsection{Categorization}
Dieses Modul bietet die Funktionalität zur Kategorisierung von Sequenzen.
Die Berechnung der Balance, Frequenz und der restlichen Sequenzwerte wird durch dieses Modul durchgeführt.
Die Logik zur Erstellung der PDF-Dateien um die berechnet Werte zu plotten, befindet sich ebenfalls in diesem Modul.
Die Funktionalität ist in Kapitel~\ref{chp:categorization} näher beschrieben.

\subsection{Cross-Correlation}
Dieses Modul dient der Berechnung der Kreuzkorrelation sowie der Suche nach Mustern in einer Sequenz mithilfe der Kreuzkorrelation.
Dafür werden unterschiedliche Funktionen sowie Einstellungsmöglichkeiten geboten, die dem Anwendungsfall angepasst werden können.
Die Logik zur Erstellung von PDF-Dateien um die in diesem Modul gewonnen Daten darzustellen, ist ebenfalls im Modul selbst enthalten.
Die Funktionalität ist in den Kapiteln~\ref{chp:crosscorrelation:implementation} und \ref{chp:crosscorrelation:patternsearch} näher beschrieben.
\chapter{Kategorisierung von Sequenzen}

\section{Kategorisierte Eigenschaften}

\subsection{Berechnung Balance}

\subsubsection{Vorgehen}

\subsubsection{Beispiele}

\subsection{Berechnung der Frequenz}

\subsubsection{Vorgehen}

\subsubsection{Beispiele}

\subsection{Auswertung der Subsequenzen}

\subsubsection{Vorgehen}

\subsubsection{Beispiele}
\chapter{Grundlagen: Kreuzkorrelation und Faltung}
Dieses Kapitel beinhaltet eine kurze Einführung zur Kreuzkorrelation und der Faltung.
Die Kreuzkorrelation wird im Projekt verwendet (siehe Kapitel~\ref{chp:crosscorrelation:implementation}), weshalb eine Einarbeitung in das Thema zum besseren Verständnis notwendig war.
Für den Fall, dass das Projekt in zukünftigen Arbeiten verwendet wird, wurde dieses Kapitel hinzugefügt um den Einstieg zu erleichtern.


\section{Einführung Kreuzkorrelation}
Die Kreuzkorrelation wird verwendet, um die Korrelation zwischen zwei Signalen oder Sequenzen zu berechnen. Dabei werden unterschiedliche Zeitverschiebungen zwischen
den Sequenzen eingesetzt. NumPy verwendet dabei folgende Formel für die 
Kreuzkorrelation\footnote{ Siehe: Dokumentation NumPy:\cite{DocumentationNumpyCorrelate} und Definition Kreuzkorrelation: \cite{DefinitionCrossCorrelation}.}.

\[
  f \star g \equiv \int\limits_{-\infty}^{+\infty} \overline{f}(-t)*g(t)
\]
Die Formel kann in eine Form umgewandelt werden, aus der die Verschiebung einer Sequenz gegenüber einer anderen besser ersichtlich 
ist\footnote{ Umformung: siehe: \cite{DefinitionCrossCorrelation}.}:
\[
  f \star g \equiv \int\limits_{-\infty}^{+\infty} \overline{f}(-\tau)*g(t + \tau) \text{ } \mathrm{d}\tau
\]

NumPy bietet dabei unterschiedliche Modi bei der Berechnung der Kreuzkorrelation an. Diese entsprechen den Modi der mathematischen Faltung ( zu Englisch \enquote{Convolution}).
Die Unterschiede der Kreuzkorrelation und Faltung sowie der Autokorrelation sind in den folgenden Abbildungen dargestellt.
Abbildung~\ref{fig:compareConvulationCrosscorrelationAutoCorrelation} bietet einen Überblick, während die Abbildungen~\ref{fig:convolutionExample1} und \ref{fig:convolutionExample2} zwei Beispiele darstellen.
Über die jeweiligen Quellenangaben können diese in animierter Form betrachtet werden.

\section{Beispiele Kreuzkorrelation, Faltung und Autokorrelation}
\begin{figure}[H]
  \centering
  \includegraphics[width=0.8\linewidth]{./images/comparisonConvolutionCorrelation.png}
  \caption[Vergleich zwischen Faltung, Kreuzkorrelation und Autokorrelation]{Vergleicht die Faltung, Kreuzkorrelation und Autokorrelation\footnotemark}
  \label{fig:compareConvulationCrosscorrelationAutoCorrelation}
\end{figure}
\footnotetext{ Quelle: \cite[]{CompareConvulationCrosscorrelationAutoCorrelation}}

\begin{figure}[H]
  \centering
  \includegraphics[width=0.8\linewidth]{./images/convolution1.png}
  \caption[Beispiel 1 für die Faltung]{Darstellung einer eindimensionalen Faltung\footnotemark}
  \label{fig:convolutionExample1}
\end{figure}
\footnotetext{ Quelle: \cite[]{ConvultionExample1Wikimedia}}

\begin{figure}[H]
  \centering
  \includegraphics[width=0.8\linewidth]{./images/convolution2.png}
  \caption[Beispiel 2 für die Faltung]{Faltung zweier Gaussfunktionen\footnotemark}
  \label{fig:convolutionExample2}
\end{figure}
\footnotetext{ Quelle: \cite[]{ConvultionExample2Wikimedia}}

\section{Bedeutung der unterschiedlichen Modi}
\begin{samepage}
  In den Beschreibungen gelten folgende Voraussetzungen: 
  \begin{align*}
    & \text{Sequenz A: von }a[0] \text{ bis }a[L_A - 1] \text{ mit } L_A = \text{Länge von Sequenz A}\\
    & \text{Sequenz A: von }b[0] \text{ bis }b[L_B - 1] \text{ mit } L_B = \text{Länge von Sequenz B}
  \end{align*}
  Als Quelle dient die Dokumentation\footnote{ Dokumentation NumPy Correlate: \cite{DocumentationNumpyCorrelate}.} sowie eine Beschreibung der 
  Modi\footnote{ Beschreibung der Modi: \cite{NumPyCorrelationModesExplained}.}.
\end{samepage}


\subsection{Modus: Valid}\label{sec:numpy_correlate_mode}
Dieser Modus wird verwendet, wenn der Modus nicht explizit angegeben wird.\\
Dabei wird eine Ausgabe der Länge $ max(L_A, L_B) - min(L_A, L_B) + 1 $ erzeugt. Die Verschiebung findet dabei im Bereich 
von $ min(L_A, L_B) - 1 \text{ bis } - max(L_A, L_B) - 1 $ statt.

\subsection{Modus: Full}
In diesem Modus wird die Berechnung im Bereich von $ 0 \text{ bis }  L_A + L_B - 2 $ durchgeführt. 
Dadurch ergibt sich eine Ausgabe von $ L_A + L_B + 1 $ Elementen. Dabei können Effekte an den Rändern der Sequenzen auftreten, da diese
sich dort nicht mehr voll überlappen.

\subsection{Modus: Same}
Liefert ein Ergebnis der Länge $ max(L_A, L_B) $ . Dabei können ebenfalls Effekte an den Rändern auftreten.\\
Die Berechnung findet im Bereich von $ \frac{(L_B - 1)}{2} \text{ bis } L_A - 1 + \frac{L_B - 1}{2} $ statt. \\
Sollte $ L_A < L_B $ gelten, werden die beiden Sequenzen vor der Berechnung ausgetauscht.











\chapter{Funktion zur Berechnung der Kreuz-Korrelation}
\label{chp:crosscorrelation:implementation}
Die Skripte zur Berechnung der Kreuz-Korrelation befinden sich im \enquote{crosscorrelation}-Verzeichnis.
Das Python Skript \enquote{functions\_crosscorrelation.py} bietet als Einstiegspunkt die Funktion\\ \mbox{\enquote{crossCorrelation(...)}} an. Im Folgenden werden die Parameter und deren Bedeutung beschrieben.

\section{Beschreibung der Parameter}

Es gibt drei Parameter, die beim Aufruf der Funktion mitgegeben werden müssen. Dabei handelt es sich um die beiden Sequenzen, die verarbeitet werden sollen sowie um die Einstellungen. Aufgrund der vielen Einstellungsmöglichkeiten, wurden diese in eigene eigene Klasse ausgelagert, die beim Aufruf als dritter Parameter übergeben werden muss.
\begin{description}[style=nextline]
\item[seqA] Die erste Datensequenz in Form eines eindimensionalen Arrays. Die enthaltenen Zahlenwerte müssen in Float konvertierbar sein.
\item[seqB] Die zweite Datensequenz in Form eines eindimensionalen Arrays. Die enthaltenen Zahlenwerte müssen in Float konvertierbar sein.
\end{description}

\subsection{Einstellungsparameter}
Folgende Parameter stehen für die Konfiguration zur Verfügung. Die standardmäßig gesetzten Werte sind mit angegeben:
\begin{description}[style=nextline]
\item[plotNormalizedData] [Default: False] Auf True setzen, wenn die beiden Sequenzen auch normalisiert dargestellt werden sollen.
\item[plotCorrelations] [Default: False] Auf True setzen, wenn die berechnete Kreuz-Korrelation dargestellt werden soll. Dabei werden alle drei Berechnungsmethoden \enquote{Valid}, \enquote{Same} und \enquote{Full} dargestellt.
\item[plotNonNormalizedResults] [Default: False] Auf True setzen, wenn auch die unnormalisierten Ergebnisse dargestellt werden sollen.
\item[plotNormalizedResults] [Default: True] Auf False setzen, wenn auch die normalisierten Ergebnisse nicht dargestellt werden sollen.
\item[subtractMeanFromResult] [Default: True] Auf False setzen, wenn das arithmethische Mittel nicht von den Ergebnissen abgezogen werden soll.
\item[drawResults] [Default: False] Wird dieser Parameter auf True gesetzt, werden die Ergebnisse auf dem Bildschirm ausgegeben. Dies kann sinnvoll sein, wenn keine PDF-Dateien erstellt werden sollen, die Ergebnisse aber dennoch sichtbar sein sollten. Der Standardwert ist auf False gesetzt, da die Skripte automatisiert durch weitere Skripte aufgerufen werden. Die Verarbeitung der Skripte blockiert dabei solange das Fenster zur Anzeige geöffnet ist. In einem automatisierten Kontext würde niemand das Fenster schließen, wodurch der Prozess blockiert werden würde.
\item[exportToPdf] [Default: False] Auf True setzen, wenn eine PDF-Datei mit den Ergebnissen generiert werden soll.
\item[exportFilePath] [Default: \enquote{}] Der Pfad zur PDF-Datei, in dem die Ergebnisse gespeichert werden sollen. Wenn \enquote{exportToPdf} auf True gesetzt wird, muss mit dieser Einstellung der Pfad zur PDF-Datei angegeben werden. Ansonsten wird die Einstellung ignoriert.
\end{description}

\section{Beispiel Ergebnisse}
Es gilt für alle Beispiele (solange nicht im Beispiel genauer spezifiziert):\\
Vorbedingung: Zwei Sequenzen mit je 10000 Einträgen. \\
SeqA:
\[ a_{n} =
  \begin{cases}
    1       & \quad \text{wenn } n \text{ enthalten in [1005,2005,1505,3505,5505,6005,6505,8005,8505,9005]}\\
    0  & \quad \text{sonst}
  \end{cases}
\]
SeqB:
\[ b_{n} =
  \begin{cases}
    1       & \quad \text{wenn } n \text{ enthalten in [1000,2000,1500,3500,5500,6000,6500,8000,8500,9000]}\\
    0  & \quad \text{sonst}
  \end{cases}
\]
Es ist ersichlich, dass die Sequenzen zueinander um einen Index von 5 verschoben sind.

\subsection{Anzeige der beiden Sequenzen mit normalisiertem Ergebnis}
Der Aufruf der Funktion mit den Standard-Parametern der Einstellungen, liefert für die Beispielsequenzen folgendes Ergebnis (dargestellt in Abbildung~\ref{fig:correlationDefaultParams}):
\begin{figure}[H]
    \includegraphics[width=\linewidth]{pythonImplementation/images/correlationDefaultParams.PNG}
    \caption[Ergebnis: Default-Parameter]{Ergebnis der Funktion bei Default-Parametern\footnotemark.}
    \label{fig:correlationDefaultParams}
\end{figure}
\footnotetext{Quelle: Eigene Darstellung}

\subsection{Auswirkungen des Parameter: plotNormalizedData}
Durch setzen des optionalen Parameters \enquote{plotNormalizedData} auf True, werden die beiden Sequenzen, zusätzlich in normalisierter Form dargestellt. 
Siehe dazu Abbildung~\ref{fig:correlationPlotNormalizedData}. 
\begin{figure}[H]
    \includegraphics[width=\linewidth]{pythonImplementation/images/correlationPlotNormalizedData.PNG}
    \caption[Ergebnis: plotNormalizedData]{Ergebnis der Funktion bei \enquote{plotNormalizedData = True} (Darstellung der Ergebnisse weggelassen)\footnotemark. }
    \label{fig:correlationPlotNormalizedData}
\end{figure}
\footnotetext{Quelle: Eigene Darstellung}

\subsection{Auswirkungen des Parameter: plotCorrelations}
Durch setzen des optionalen Parameters \enquote{plotCorrelations} auf True, 
wird die Kreuz-Korrelation der beiden Sequenzen berechnet.
Dabei werden alle drei Berechnungsmethoden \enquote{Valid}, \enquote{Same} und \enquote{Full} dargestellt.
Siehe dazu Abbildung~\ref{fig:correlationPlotCorrelations}. 
\begin{figure}[H]
    \includegraphics[width=\linewidth]{pythonImplementation/images/correlationPlotCorrelations.PNG}
    \caption[Ergebnis: plotCorrelations]{Zusätzliche Ausgabe der Funktion bei \enquote{plotCorrelations = True} (Darstellung der Ergebnisse weggelassen)\footnotemark. }
    \label{fig:correlationPlotCorrelations}
\end{figure}
\footnotetext{Quelle: Eigene Darstellung}

\subsection{Auswirkungen des Parameter: plotNonNormalizedResults}
Durch setzen des optionalen Parameters \enquote{plotNonNormalizedResults} auf True, können die Ergebnisse zusätzlich in unnormalisierter Form ausgegeben werden.
Siehe dazu Abbildung~\ref{fig:correlationPlotNonNormalizedResults}. 
\begin{figure}[H]
    \includegraphics[width=\linewidth]{pythonImplementation/images/correlationPlotNonNormalizedResults.PNG}
    \caption[Ergebnis: plotNonNormalizedResults]{Zusätzliche Ausgabe der Funktion bei \enquote{plotNonNormalizedResults = True}\footnotemark. }
    \label{fig:correlationPlotNonNormalizedResults}
\end{figure}
\footnotetext{Quelle: Eigene Darstellung}

\subsection{Auswirkungen des Parameter: plotNormalizedResults}
Durch setzen des optionalen Parameters \enquote{plotNormalizedResults} auf False, kann die Ausgabe der normalisierten Ergebnisse verhindert werden.

\subsection{Auswirkungen des Parameter: subtractMeanFromResult}
Um die Auswirkung des Parameters darzustellen, werden folgende Sequenzen vorrausgesetzt (je 10000 Einträge):\\
SeqA:
\[ a_{n} =
  \begin{cases}
    2       & \quad \text{wenn } n \text{ enthalten in [1005,2005,1505,3505,5505,6005,6505,8005,8505,9005]}\\
    1  & \quad \text{sonst}
  \end{cases}
\]
SeqB:
\[ b_{n} =
  \begin{cases}
    2       & \quad \text{wenn } n \text{ enthalten in [1000,2000,1500,3500,5500,6000,6500,8000,8500,9000]}\\
    1  & \quad \text{sonst}
  \end{cases}
\]
Die Sequenzen sind wieder um einen Wert von 5 verschoben, allerdings wechseln diese nicht zwischen 0 und 1, sondern zwischen 1 und 2. 
Die Ergebnisse können dabei unübersichtlich werden, deshalb wird ohne Angabe des Parameters, True als Standardwert verwendet.
Dadurch wird das arithmethische Mittel des Ergebnis vom Ergebnis selbst abgezogen. Abbildung~\ref{fig:correlationSubtractMeamFromResultTrue} 
und \ref{fig:correlationSubtractMeamFromResultFalse} zeigen die Auswirkung des Parameters. 
\begin{figure}[H]
    \includegraphics[width=\linewidth]{pythonImplementation/images/correlationSubtractMeamFromResultFalse.PNG}
    \caption[Ergebnis: subtractMeanFromResult = False]{Darstellung des Ergebnis bei \enquote{subtractMeanFromResult = False}. Ausschläge sind sehr schwer erkennbar\footnotemark. }
    \label{fig:correlationSubtractMeamFromResultFalse}
\end{figure}
\footnotetext{Quelle: Eigene Darstellung}

\begin{figure}[H]
    \includegraphics[width=\linewidth]{pythonImplementation/images/correlationSubtractMeamFromResultTrue.PNG}
    \caption[Ergebnis: subtractMeanFromResult = True]{Darstellung des Ergebnis bei \enquote{subtractMeanFromResult = True}. Ausschläge sind besser erkennbar\footnotemark. }
    \label{fig:correlationSubtractMeamFromResultTrue}
\end{figure}
\footnotetext{Quelle: Eigene Darstellung}


\section{Beispiel-Skripte}
Ein Skript welches die Funktion aus diesem Kapitel nutzt ist als Beispiel im Projekt unter \enquote{example\_script\_crosscorrelation.py} abgelegt.\\
Das Skript das im Zuge der Automatisierung verwendet wird, ist im Unterverzeichnis \enquote{crosscorrelation} unter \enquote{automated\_crosscorrelation.py} zu finden.









\chapter{Suche nach Pattern mit Hilfe der Kreuz-Korrelation}
\label{chp:crosscorrelation:patternsearch}
Das Python Skript \enquote{functions\_crosscorrelation\_patternsearch.py} bietet zwei Funktionen an, die verwendet werden können, 
um nach einem gegebenen Pattern in einer Sequenz zu suchen.

\section{Angebotene Funktionen}
Folgende Funktionen werden angeboten:

\subsection{Funktion: getCorrelationDataForPatternSearch}
Mit Hilfe der \enquote{getCorrelationDataForPatternSearch} Funktion kann nach einem Pattern in einer gegebenen Sequenz gesucht werden. Dafür sind folgende Parameter nutzbar:
\begin{description}
    \item[seqA] Die Sequenz in der nach dem Pattern gesucht werden soll. Übergeben in Form eines eindimensionalen Arrays. Die enthaltenen Zahlenwerte müssen in Float konvertierbar sein.
    \item[pattern] Das Pattern nach dem gesucht werden soll. Übergeben in Form eines eindimensionalen Arrays. Die enthaltenen Zahlenwerte müssen in Float konvertierbar sein.
    \item[plotCorrelation] [Default: False] Auf True setzen, wenn die berechnete Kreuz-Korrelation dargestellt werden soll.
\end{description}
Die Funktion gibt die berechnete Kreuz-Korrelation zurück, die in der nächsten Funktion \enquote{extractIndicesFromCorrelationData} verwendet werden kann.
\\
\\
Die Funktion verwendet intern die von NumPy bereitgestellte \enquote{correlate} Funktion. Als Modus wird der in Kapitel~\ref{sec:numpy_correlate_mode} beschriebene Modus \enquote{Valid} genutzt.

\subsection{Funktion: extractIndicesFromCorrelationData}
Die Funktion kann verwendet werden, um die Indizes zu berechnen, an denen das Pattern in der Sequenz auftritt. Dafür muss der Funktion das Ergebnis der \enquote{getCorrelationDataForPatternSearch} Funktion
übergeben werden. Als Rückgabe erhält man die Indizes und der Wert der Korrelation an dieser Stelle.\\
Optional kann ein Schwellwert übergeben werden. Je nachdem wie dieser Schwellwert gewählt ist, werden mehr oder weniger Indizes zurückgegeben. Bei einem hohen Schwellwert wird 
jeweils nur der erste Index zurückgegeben, bei dem das Pattern auftritt. Bei einem niedriger gewählten Schwellwert werden alle Indizes zurückgegeben, an denen das Pattern auftritt. 
Bei einem zu niedrigem Schwellwert werden auch Indizes zurückgegeben, die außerhalb des eigentlichen Patterns liegen. Diese Bereiche sind direkt vor und nach dem Ende des Patterns. Ein zu niedriger Schwellwert kann erkannt werden, wenn der gefundene Bereich größer als das Pattern selbst ist. Durch einen höheren Schwellwert kann dieses \enquote{Auswaschen} oder \enquote{Aufweichen} verhindert werden.

\subsection{Beispiel für die Pattern-Suche}
Vorbedingung: Eine Sequenzen mit 10000 Einträgen. \\
SeqA:
\[ a_{n} =
  \begin{cases}
    1       & \quad \text{wenn } n \text{ enthalten in [1005,1007,1010,6005,6007,6010]}\\
    0  & \quad \text{sonst}
  \end{cases}
\]
Gesuchtes Pattern:
\[
pattern = [1,0,1,0,0,1,0] 
\]
Das Pattern tritt in der gegeben Sequenz zwei Mal auf (an Index 1005 und 6005).
Der Aufruf der \enquote{getCorrelationDataForPatternSearch} Funktion mit dem \enquote{plotCorellation = True} zeigt die in Abbildung~\ref{fig:patternsearchCorrelation} dargestellte Kreuz-Korrelation.

\begin{figure}[H]
  \includegraphics[width=\linewidth]{./images/patternsearchCorrelation.PNG}
  \caption[Patternsuche: Kreuz-Korrelation]{Kreuz-Korrelation der Patternsuche\footnotemark.}
  \label{fig:patternsearchCorrelation}
\end{figure}
\footnotetext{Quelle: Eigene Darstellung}

\begin{samepage}
Mit Hilfe der dargestellten Kreuz-Korrelation kann der Schwellwert für die \enquote{extractIndicesFromCorrelationData} festgelegt werden.
Ein Aufruf mit einem Schwellwert von \enquote{0.3} liefert folgendes Ergebnis: \\
\begin{align*}
   & [6005, 0.70710678], \\
   & [1005, 0.70710678]
\end{align*}

Dabei handelt es sich um die Indizes, an denen das Pattern in der Sequenz beginnt (inklusive den Wert an dieser Stelle). \\
\end{samepage}
\begin{samepage}
Ein Aufruf mit einem Schwellwert von \enquote{0.2} liefert dagegen folgendes Ergebnis: \\
\begin{align*}
  &  [6005, 0.70710678], \\
  & [1005, 0.70710678], \\
  & [6003, 0.23570226], \\
  & [1000, 0.23570226], \\
  & [6007, 0.23570226], \\
  & [6010, 0.23570226], \\
  & [6002, 0.23570226], \\
  & [6000, 0.23570226], \\
  & [1010, 0.23570226], \\
  & [1008, 0.23570226], \\
  & [1007, 0.23570226], \\
  & [1003, 0.23570226], \\
  & [1002, 0.23570226], \\
 & [6008, 0.23570226]
\end{align*}
Darin enthalten sind Indizes im Bereich von 1000 bis 1010 sowie 6000 bis 6010. Die Bereiche sind größer als das Pattern selbst, allerdings ist das Pattern in diesen Bereichen enthalten.
\end{samepage}


\chapter{Automatische Ausführung der Skripte}

\section{Bereitgestellte Skripte}
\textcolor{red}{VS-Code + CMD Anleitung + Darstellung der Ergebnisse von Beispielsequenzen + Logging}

\section{Anleitung zur Ausführung Skripte}

\section{Ablauf der automatisierten Ausführung}

\section{Verbesserungsmöglichkeiten und Erweiterungen}
\textcolor{red}{Multi-Prozess PDF Generierung}