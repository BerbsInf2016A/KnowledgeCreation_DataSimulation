\section{Aufbau des Python Projekts}
Der Quellcode ist nach der jeweiligen Funktionalität in eigene Module aufgeteilt.
Dadurch ist es auch möglich, nur einzelne Funktionalitäten in eigenen Skripten zu verwenden.
Die Module selbst sind wiederum in einzelne Dateien gegliedert.
Dadurch werden die Aufgaben und Funktionalitäten noch weiter aufgeteilt, um die Lesbarkeit des Codes zu erhöhen und die Komplexität zu reduzieren.
Zusätzlich zur Dokumentation in dieser Arbeit, wurde der Code kommentiert.

\subsection{Top-Level Ansicht}
Im Oberverzeichnis befinden sich die folgenden Dateien und Ordner. Dieses Verzeichnis dient als Einstiegspunkt zur Ausführung der Skripte.
\begin{description}[style=nextline]
	\item[Skriptdateien] Der Einstiegspunkt zur automatisierten Ausführung bildet das Skript in \enquote{automated\_script.py}. Zum Zweck der Dokumentation sind weitere Beispielskripte abgelegt, die das Prefix \enquote{example} im Dateinamen tragen.
	\item[Unterordner] In den Unterordnern \enquote{analysationrequest} \enquote{categorization} und \enquote{crosscorrelation} sind die Skripte für den jeweiligen Bereich abgelegt. Diese werden in den folgenden Sektionen genauer beschrieben. Eine Ausnahme bildet der \enquote{sourceFiles} Ordner. Dieser Ordner wird für die Quelldateien verwendet, wenn bei der Ausführung kein alternativer Pfad angegeben wird. Sollte der Ordner nicht existieren, so wird dieser automatisch erstellt.
	\item[Docker-File] Diese Datei kann verwendet werden, um einen Docker-Container zu erstellen. Siehe dazu die notwendigen Befehle im Anhang unter ~\ref{docker:examples}.
	\item[Dependencies.txt] In dieser Datei ist der Befehl zur Installation der notwendigen Pakete hinterlegt.
\end{description}

\subsection{Analysation-Request}
Das Modul \enquote{AnalysationRequest} bildet die Grundlage der Analysen.
Für jede Quelldatei wird ein Request angelegt, der anschließend durch die weiteren Module verarbeitet wird.
Die Ergebnisse der jeweiligen Module werden ebenfalls im Request festgehalten.
Dieses Modul beinhaltet auch die Logik und Funktionalität zum Einlesen der Quelldateien, sowie zur Expansion der komprimierten Quelldaten.


\subsubsection{Validierung des Datenformats}
Eingelesen werden können Dateien die dem in Kapitel~\ref{csvMoreFiles} beschriebenen komprimierten Datenformat entsprechen.
Dabei werden die angegebenen Werte überprüft.
Das Format schreibt vor, dass ein einzelner Eintrag die Anzahl und den Wert der Sequenz für diesen Block hat.
Wenn die Anzahl nicht größer 0 ist, wird eine entsprechende Fehlermeldung ausgegeben.
Auch der Wert wird dahingehend überprüft, dass dieser nur 0 oder 1 sein kann.
Angaben die keine Zahlen sind werden nicht akzeptiert. 

\subsection{Categorization}
Dieses Modul bietet die Funktionalität zur Kategorisierung von Sequenzen.
Die Berechnung der Balance, Frequenz und der restlichen Sequenzwerte wird durch dieses Modul durchgeführt.
Die Logik zur Erstellung der PDF-Dateien um die berechnet Werte zu plotten, befindet sich ebenfalls in diesem Modul.
Die Funktionalität ist in Kapitel~\ref{chp:categorization} näher beschrieben.

\subsection{Cross-Correlation}
Dieses Modul dient der Berechnung der Kreuzkorrelation sowie der Suche nach Mustern in einer Sequenz mithilfe der Kreuzkorrelation.
Dafür werden unterschiedliche Funktionen sowie Einstellungsmöglichkeiten geboten, die dem Anwendungsfall angepasst werden können.
Die Logik zur Erstellung von PDF-Dateien um die in diesem Modul gewonnen Daten darzustellen, ist ebenfalls im Modul selbst enthalten.
Die Funktionalität ist in den Kapiteln~\ref{chp:crosscorrelation:implementation} und \ref{chp:crosscorrelation:patternsearch} näher beschrieben.