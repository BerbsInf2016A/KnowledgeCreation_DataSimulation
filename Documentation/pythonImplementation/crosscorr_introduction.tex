\chapter{Grundlagen: Kreuzkorrelation und Faltung}
Dieses Kapitel beinhaltet eine kurze Einführung zur Kreuzkorrelation und der Faltung.
Die Kreuzkorrelation wird im Projekt verwendet (siehe Kapitel~\ref{chp:crosscorrelation:implementation}), weshalb eine Einarbeitung in das Thema zum besseren Verständnis notwendig war.
Für den Fall, dass das Projekt in zukünftigen Arbeiten verwendet wird, wurde dieses Kapitel hinzugefügt um den Einstieg zu erleichtern.


\section{Einführung Kreuzkorrelation}
Die Kreuzkorrelation wird verwendet, um die Korrelation zwischen zwei Signalen oder Sequenzen zu berechnen. Dabei werden unterschiedliche Zeitverschiebungen zwischen
den Sequenzen eingesetzt. NumPy verwendet dabei folgende Formel für die 
Kreuzkorrelation\footnote{ Siehe: Dokumentation NumPy:\cite{DocumentationNumpyCorrelate} und Definition Kreuzkorrelation: \cite{DefinitionCrossCorrelation}.}.

\[
  f \star g \equiv \int\limits_{-\infty}^{+\infty} \overline{f}(-t)*g(t)
\]
Die Formel kann in eine Form umgewandelt werden, aus der die Verschiebung einer Sequenz gegenüber einer anderen besser ersichtlich 
ist\footnote{ Umformung: siehe: \cite{DefinitionCrossCorrelation}.}:
\[
  f \star g \equiv \int\limits_{-\infty}^{+\infty} \overline{f}(-\tau)*g(t + \tau) \text{ } \mathrm{d}\tau
\]

NumPy bietet dabei unterschiedliche Modi bei der Berechnung der Kreuzkorrelation an. Diese entsprechen den Modi der mathematischen Faltung ( zu Englisch \enquote{Convolution}).
Die Unterschiede der Kreuzkorrelation und Faltung sowie der Autokorrelation sind in den folgenden Abbildungen dargestellt.
Abbildung~\ref{fig:compareConvulationCrosscorrelationAutoCorrelation} bietet einen Überblick, während die Abbildungen~\ref{fig:convolutionExample1} und \ref{fig:convolutionExample2} zwei Beispiele darstellen.
Über die jeweiligen Quellenangaben können diese in animierter Form betrachtet werden.

\section{Beispiele Kreuzkorrelation, Faltung und Autokorrelation}
\begin{figure}[H]
  \centering
  \includegraphics[width=0.8\linewidth]{./images/comparisonConvolutionCorrelation.png}
  \caption[Vergleich zwischen Faltung, Kreuzkorrelation und Autokorrelation]{Vergleicht die Faltung, Kreuzkorrelation und Autokorrelation\footnotemark}
  \label{fig:compareConvulationCrosscorrelationAutoCorrelation}
\end{figure}
\footnotetext{ Quelle: \cite[]{CompareConvulationCrosscorrelationAutoCorrelation}}

\begin{figure}[H]
  \centering
  \includegraphics[width=0.8\linewidth]{./images/convolution1.png}
  \caption[Beispiel 1 für die Faltung]{Darstellung einer eindimensionalen Faltung\footnotemark}
  \label{fig:convolutionExample1}
\end{figure}
\footnotetext{ Quelle: \cite[]{ConvultionExample1Wikimedia}}

\begin{figure}[H]
  \centering
  \includegraphics[width=0.8\linewidth]{./images/convolution2.png}
  \caption[Beispiel 2 für die Faltung]{Faltung zweier Gaussfunktionen\footnotemark}
  \label{fig:convolutionExample2}
\end{figure}
\footnotetext{ Quelle: \cite[]{ConvultionExample2Wikimedia}}

\section{Bedeutung der unterschiedlichen Modi}
\begin{samepage}
  In den Beschreibungen gelten folgende Voraussetzungen: 
  \begin{align*}
    & \text{Sequenz A: von }a[0] \text{ bis }a[L_A - 1] \text{ mit } L_A = \text{Länge von Sequenz A}\\
    & \text{Sequenz A: von }b[0] \text{ bis }b[L_B - 1] \text{ mit } L_B = \text{Länge von Sequenz B}
  \end{align*}
  Als Quelle dient die Dokumentation\footnote{ Dokumentation NumPy Correlate: \cite{DocumentationNumpyCorrelate}.} sowie eine Beschreibung der 
  Modi\footnote{ Beschreibung der Modi: \cite{NumPyCorrelationModesExplained}.}.
\end{samepage}


\subsection{Modus: Valid}\label{sec:numpy_correlate_mode}
Dieser Modus wird verwendet, wenn der Modus nicht explizit angegeben wird.\\
Dabei wird eine Ausgabe der Länge $ max(L_A, L_B) - min(L_A, L_B) + 1 $ erzeugt. Die Verschiebung findet dabei im Bereich 
von $ min(L_A, L_B) - 1 \text{ bis } - max(L_A, L_B) - 1 $ statt.

\subsection{Modus: Full}
In diesem Modus wird die Berechnung im Bereich von $ 0 \text{ bis }  L_A + L_B - 2 $ durchgeführt. 
Dadurch ergibt sich eine Ausgabe von $ L_A + L_B + 1 $ Elementen. Dabei können Effekte an den Rändern der Sequenzen auftreten, da diese
sich dort nicht mehr voll überlappen.

\subsection{Modus: Same}
Liefert ein Ergebnis der Länge $ max(L_A, L_B) $ . Dabei können ebenfalls Effekte an den Rändern auftreten.\\
Die Berechnung findet im Bereich von $ \frac{(L_B - 1)}{2} \text{ bis } L_A - 1 + \frac{L_B - 1}{2} $ statt. \\
Sollte $ L_A < L_B $ gelten, werden die beiden Sequenzen vor der Berechnung ausgetauscht.










