\chapter{Einleitung}
\label{chp:introduction}
Industriemaschinen erheben heutzutage eine große Menge an Daten.  Zum einen versprechen sich Unternehmen einen Gewinn an Produktivität durch eine Entwicklung zur Industrie 4.0, zum anderen ist es durch günstige Sensorik überhaupt erst möglich geworden Maschinendaten zu erfassen\footnote{https://www.scope-online.de/smart-industry/kuenstliche-intelligenz-in-der-smart-factory.htm}. So werden bspw. Informationen zu einzelnen Produktionsmaschinen, bis hin zu ganzen Maschinengruppen gespeichert.
 
Viele Unternehmen stehen vor dem Problem diese Daten mit geeigneten Methoden und Werkzeugen zu analysieren. Die Ergebnisse solcher Analysen lassen sich allerdings für die Optimierung von Produktionsprozessen nutzen.

In dieser Arbeit werden binäre Datenstrukturen definiert und konstruiert, um mit deren Hilfe Information aus Realdaten aus Produktionsprozessen zu gewinnen. Dabei gilt es, interpretierbare Charakteristika, Regelmäßigkeiten und Gesetzmäßigkeiten in den Daten zu suchen.

\textcolor{red}{Not finished}

\textcolor{red}{Zeitliche Korrelation zwischen Sequenzen}

\textcolor{red}{
	Allgemeines Vorgehen anhand eines Realsbeispiels:
	\begin{list}{}{}
		\item Rohdaten vorbereiten
		\item Kategorisierung der Daten 
		\item Verschiebung von Sequenzen zueinander
		\item Erkennung von Muster (auch mit Teilsequenzen)
		\item Nach Mustern in anderen Sequenzen suchen
	\end{list}
}

\textcolor{red}{
	Hypothesen:
	\begin{list}{}{}
		\item  
	\end{list}
}