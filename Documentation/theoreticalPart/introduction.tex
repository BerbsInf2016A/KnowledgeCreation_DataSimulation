\chapter{Einleitung}
\label{chp:introduction}
Industriemaschinen erheben heutzutage eine große Menge an Daten. Zum einen versprechen sich Unternehmen einen Gewinn an Produktivität durch eine Entwicklung zur Industrie 4.0, zum anderen ist es durch günstige Sensorik überhaupt erst möglich geworden Maschinendaten in diesem Umfang zu erfassen\footnote{ \cite{SmartFactory}}.

Mit Hilfe von Methoden und Werkzeugen des Data-Mining lassen sich diese Daten analysieren und so neue Zusammenhänge erkennen. Dies soll Unternehmen bei der zukünftigen Planung und Ausführung von Produktionsprozessen helfen.

Diese Arbeit beschäftigt sich, parallel zu weiteren Arbeiten des Studienganges Angewandte Informatik der DHBW Mosbach, mit der Grundlagenforschung in diesem Bereich. Mit Hilfe simulierter Daten soll der Aufbau sowie Eigenschaften von Produktionsdaten erforscht werden. Ziel dabei ist es, Sequenzen zu Charakterisieren und Klassifizieren.

Zu Beginn werden Sequenzen als binäre Datenstrukturen eingeführt. Diese bestehen nur aus den Zuständen \textit{Fehler} und \textit{kein Fehler}. Neben einer formalen Beschreibung werden die Eigenschaften solcher Sequenzen vorgestellt. Auf Grundlage dieser Eigenschaften soll es möglich sein, Sequenzen zu klassifizieren oder untereinander zu vergleichen. In einem beispielhaften Vorgehen werden die Möglichkeiten der Analyse auf Grundlage der theoretischen Inhalte herausgearbeitet.

Im zweiten Teil dieser Arbeit wird auf die technische Umsetzung einer automatisierten Auswertung von Sequenzen eingegangen. Dazu wird als erstes Beschrieben wie sich Daten in ein einheitliches und vergleichbares Format konvertieren lassen. Zudem wird ein Tool zur Generierung von Simulationsdaten, welches im Rahmen dieser Studienarbeit implementiert wurde, vorgestellt. Daran anschließend wird die Umsetzung eines weiteren Tools, zur automatisierten Analyse von simulierten Daten oder Produktionsdaten, erläutert. 