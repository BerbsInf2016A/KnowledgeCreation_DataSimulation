\chapter{Einleitung}
\label{chp:introduction}
Industriemaschinen erheben heutzutage eine große Menge an Daten. Zum einen versprechen sich Unternehmen einen Gewinn an Produktivität durch eine Entwicklung zur Industrie 4.0, zum anderen ist es durch günstige Sensorik überhaupt erst möglich geworden Maschinendaten zu erfassen\footnote{https://www.scope-online.de/smart-industry/kuenstliche-intelligenz-in-der-smart-factory.htm}. So werden bspw. Informationen zu einzelnen Produktionsmaschinen, bis hin zu ganzen Maschinengruppen gespeichert. Viele Unternehmen stehen vor dem Problem diese Daten mit geeigneten Methoden und Werkzeugen zu analysieren.

In dieser Arbeit werden binäre Datenstrukturen definiert und konstruiert, um mit deren Hilfe Information aus Realdaten aus Produktionsprozessen zu gewinnen. Dabei gilt es, interpretierbare Charakteristika, Regelmäßigkeiten oder Gesetzmäßigkeiten in den Daten zu suchen. 

Zu Beginn werden dazu Sequenzen als binäre Datenstrukturen eingeführt. Solche Sequenzen bestehen nur aus den Zuständen \textit{Fehler} und \textit{kein Fehler}. Neben einer formalen Beschreibung werden die Eigenschaften solcher Sequenzen vorgestellt. Daneben werden Kennzahlen und Muster zur Klassifizierung definiert. In einem beispielhaften Vorgehen werden die Möglichkeiten der Analyse auf Grundlage der theoretischen Inhalte herausgearbeitet.

Im zweiten Teil dieser Arbeit wird auf die technische Umsetzung einer automatisierten Auswertung von Sequenzen eingegangen. \textcolor{red}{genauer beschreiben...}  

Den aktuellen Stand der Wissenschaft und bisher geleistete Arbeit in diesem Gebiet sind nicht einfach herauszuarbeiten. Besonders im Bereich der Wissensgenerierung aus Produktionsdaten gestaltet es sich schwer Grundlagenliteratur zu finden. Ein Großteil der Veröffentlichungen beschäftigt sich auf einem höheren Level mit diesem Thema. Dagegen gibt es mehr wissenschaftliche Grundlagenliteratur im Bereich der Bioinformatik, die sich mit diesem Thema auseinandersetzt. Unter dem Stichwort \textit{Sequenzanalyse} findet man bspw. Algorithmen zur Analyse von DNA oder RNA. 